%!TEX root = ../thesis.tex
\section{Motivation}

In recent years the web has become a ubiquitous platform\footnote{More than 3/4 of the people in developed countries have access to the Internet both via a computer and via a mobile device. \cite{ITU2013}} which is used by many on a daily basis. Its usage has changed dramatically over time and has developed from primarily being a source of information to being a platform that is used for social activities, entertainment and productive work. This change has been driven by standardized technological innovation of the browsers vendors, which allow developers to build more complex and more interactive applications. In the last few years for example, APIs for advanced network access, APIs for peer-to-peer communication and APIs for 3D rendering have been standardized and been introduced in modern web browsers.

Modern web applications very often allow its users to collaborate and to interact with one another. Either by allowing social interactions like messaging other users or to comment on user-generated content or by allowing users to work together collaboratively like in Google Docs. Especially the case of Google Docs has shown how powerful collaborative work on the web can be. Collaborative web applications benefit from many years of research and experience in desktop-based synchronization algorithms and more and more applications are starting to introduce these algorithms.

A new addition to the list of novel APIs is the Web Audio API which gives access to advanced low-level audio features that can be used to precisely compose music and to produce sounds synthetically. Since the API had been introduced in the first browsers, a lot of interactive sound experiments had been created that showed what kind of applications are now possible to create by only using web technology. The API is still marked as being an early draft, but its web browser implementations are already stable enough to build applications on top of it.

The purpose of this thesis is to build a prototypical collaborative web-based Digital Audio Workstation (DAW) on top of the Web Audio API and on top of new communication APIs which serve as a synchronization channel. Digital Audio Workstations have existed for many years as desktop application and they allow users to create songs entirely digital. This thesis acts as a test to try out the limitations of the Web Audio API and the web as a platform for a replacement or as an addition to existing DAW applications. Furthermore, it is a test to see how good collaboration works in DAWs, because most DAWs do not feature collaborative editing.

\section{Disclaimer}

At the time the work on this thesis started, the above mentioned approach was a complete novelty. But during the development phase the software `Soundtrap'\footnote{\url{https://www.soundtrap.com/}, accessed at 28.03.2014} by `Playwerk AB' has been released in an early alpha stage. It is a similar product to the application that will be described in this thesis. Soundtrap, however, is based on a completely different technology stack and algorithms. Additionally, it is not sure if the collaborative feature of that editor uses one of the enhanced synchronization algorithms described here.

\section{Structure}

The first part of this thesis (\refchapter{part:theory}) contains the theoretical background that is needed in order to implement collaborative audio editing software. \refchapter{ch:audio-theory} dives into the backgrounds of digital signal processing and how it can be realized in modern web applications. \refchapter{ch:sync-theory} then shows different concepts of data synchronization and discusses which approach is most suitable for the application of this thesis. \refchapter{ch:realtime-theory} lists and describes the available communication technologies on the web and examines, which of the presented technologies is most suitable for the chosen synchronization model.

The second part (\refchapter{part-concept}) explains the concept of the application. In \refchapter{ch:concept-editor} all the editor's components are listed. Furthermore, it is explained why exactly these components were chosen and how they compare to the components in other DAWs.

The third part (\refchapter{part:implementation}) is about how the application that was described in \refchapter{part-concept} has been implemented. \refchapter{ch:impl-editor} goes into the implementation details and techniques of each of the editor's components. The web application that was built around it is explained in \refchapter{ch:impl-backend}. The implementation of the synchronization algorithm and the enhancements that were made during this thesis are explained in \refchapter{impl-sync-algorithm}.

The fourth and last part of thesis (\refchapter{part:conclusion}) then evaluates (\refchapter{ch:evaluation}) the application prototype and the state of audio processing on the web and gives a future outlook on further improvements and additions that will be made to the editor (\refchapter{ch:outlook}).