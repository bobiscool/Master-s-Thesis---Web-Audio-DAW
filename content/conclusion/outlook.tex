%!TEX root = ../../thesis.tex
\chapter{Outlook}
\label{ch:outlook}

This thesis has presented a prototypical version of a collaborative music editor. In order to turn the prototype into a production-ready tool, it would need more features and existing features would have to be enhanced.

\subsection{Advanced Editing Features}

Currently, the editor does not support keyboard commands to, e.g., control playback or to manipulate notes and pieces. Modern DAWs all support a range of keyboard commands which highly speed up the creative process and give musicians more control. Another important aspect would be to add undo functionality. Since all operations are synced immediately to all other participants, the undo function could be tricky to implement. However, collaborative undo operations have been implemented in Operational Transformation systems before (e.g. by \cite{ferrie2004concurrentundo}) and the concepts seem to be translatable to Differential Synchronization. In addition to new operations, the usability of the editor could be enhanced by introducing a snapping functionality for all draggable elements, so that it will be much easier to correctly arrange pieces and notes.

\subsection{Export}

For now, the only way to export an arrangement is to record it and download the resulting WAV file. The WAV format however, is not a common file format for the distribution of songs. The most common format is MP3. There are plenty of MP3 encoders available for free usage and the editor could encode the file for the musician after a WAV file has been generated. Additionally, a more advanced export to platforms like Soundcloud\footnote{\url{http://soundcloud.com}} and Mixcloud\footnote{\url{http://mixcloud.com}} could be added so that musicians do not have to upload their songs manually. Both services provide APIs that allow third-party applications to upload new songs and they are very famous amongst professional and hobby musicians.

\subsection{Web MIDI API}

MIDI, short for Musical Instrument Digital Interface, is a protocol that is used to connect digital instruments (e.g. digital musical keyboards) to computers and to transfer the played notes and settings in a structured way \cite[p. 972ff]{curtis1996computer}. The standard is very widespread and almost all music applications and digital instruments support it. Implementing the MIDI protocol in JavaScript is, of course, possible but it is not possible to communicate with connected MIDI devices. Adding the ability to play and change the settings of the synthesizer with a connected keyboard could enrich the user's experience and also make creating synthesizer notes much easier, because chords would not need to be created one note at a time, but simultaneously. For this reason, the Web MIDI API has been created\footnote{\url{http://www.w3.org/TR/webmidi/}}. It provides a full JavaScript implementation of the MIDI protocol and helper functions to communicate to connected MIDI devices. The API is still in a very early development stage and therefore not available in most web browsers\footnote{\url{http://caniuse.com/midi}, last checked on 21/03/2014}. In fact, it is only available in Chrome when an experimental flag is activated. Once the API is more widely available, the editor can be enhanced to also support MIDI instruments.

\subsection{Standalone application}

The evaluation (see \refchapter{ch:evaluation}) has shown that one major drawback of the application, compared to other editors, is that it needs constant Internet access to save the changes and to upload files to the server. The synchronization algorithm however, is designed to be able to also merge versions that have diverged for a longer amount of time. In theory it should be able to also merge versions of documents that have diverged while a client was offline for a longer time. Further research needs to be done on that topic, but if the algorithm, or an altered version thereof, proves to be suitable for this purpose, a standalone version of the editor could be created. Standalone means that the frontend application of the editor would be wrapped into a native application (probably using a tool like node-webkit\footnote{\url{https://github.com/rogerwang/node-webkit}, last checked on 21/03/2014}) and it would still communicate to a central server for synchronization. Once the client is offline, all changes will be stored locally and then synced with the server when the client comes back online. The same mechanism could be used for file uploads. In this way, the editor could compete more with existing DAWs on the market.

\subsection{Effects}

Audio effects were introduced in \refchapter{sec:webaudio-effects} but there was no time to implement them. Effects are an essential building block of all DAWs and the lack of effects in the editor is one of the biggest drawbacks at the moment. Nonetheless, it has been shown that implementing effects in the Web Audio API is possible and implementing the drafted user interface is just a matter of time. There are already a variety of effects implemented in the previously mentioned tuna.js\footnote{\url{https://github.com/Dinahmoe/tuna}, last checked on 20/03/2014}. A first look at its API showed that it would be possible to use the plugin with the synchronization library and the editor's node graph.

\subsection{More complex instruments}

The initial set of `instruments' was developed for the purpose of creating a first prototype, but for production-ready usage they should be enriched by additional features. The drum machine for example, could be enhanced by adding more drumsets or a `swing' mode, which plays the individual notes a bit out of sync to create a more variably sounding drum loop. Also, a library of preset drum patterns could be added from which the user could then choose. Several other DAWs already ship with many predefined drum patterns that users can use for free. An enhancement for the synthesizer could be the addition of a `patch' system. Patches are synthesizer setups that can be stored and loaded, so that it is easier to reuse setups in different projects or even in the same project.

\subsection{Workspace Awareness}

Currently, there is no way for users to see which parts of an arrangement are currently being edited by other users. Though this is not a feature which has a negative influence on the performance of the synchronization algorithm, it is important for users to see what other users are editing simultaneously. \cite[p. 5f]{koren2013sharedediting} list several possible awareness modules like a participant list or user-specific colors. User-specific colors in their context only means that text parts have a special background color, dependent on the user who edited these parts. In the context of this thesis's audio editor, it could be used for border colors of the elements that are currently being edited by a user.